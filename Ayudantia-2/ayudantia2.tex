\documentclass[12pt]{article}
\usepackage{amsmath} % Necesario para el comando \dfrac
\usepackage{amssymb}
\usepackage{graphicx} % Required for inserting images
\usepackage{enumitem} % Required for customizing enumeration
\usepackage{mathtools} % Necesario para PVI's con funciones por tramos
\usepackage{textcomp}
\usepackage[left=2cm,right=2cm,top=2cm,bottom=2cm]{geometry}
\usepackage[scr]{rsfso}
\usepackage{listings}
\usepackage{hyperref}
\usepackage{multicol}
\everymath{\displaystyle}

\begin{document}

    \begin{titlepage}
        \centering
        \includegraphics[width=0.3\textwidth]{../imgs/logo-uai-fic.png}
        
        \vspace{0.5cm}
        \textbf{\fontsize{12}{24} Ayudantía 2 Estructura de datos}
        
        \vspace{0.5cm}
        \textbf{\fontsize{12}{24}\selectfont Profesores: Sebastián Sáez, Diego Ramos}
        
        \begin{center}
            \textbf{\fontsize{12}{24}\selectfont Ayudantes: Diego Duhalde, Benjamín Wiedmaier, Fernando Zamora}
        \end{center}

        \textbf{\fontsize{12}{24} Preparación para Prueba 1}

        \begin{enumerate}
            \item Lenguaje C
            \begin{enumerate}[label*=\arabic*.]
                \item ¿Qué es el operador \texttt{sizeof}? ¿Cómo se utiliza?
                \item ¿Qué es la aritmética de punteros? Proporcione un ejemplo.
                \item Explique la diferencia entre pasar un argumento por valor y pasar un argumento por referencia.
                \item Explique la diferencia entre una variable local y una variable global.
                \item ¿Cual es la diferencia entre \& y * en C?
                \item Indique la diferencia central entre la memoria estática (stack) y la memoria dinámica (heap).
            \end{enumerate}    
            \item Punteros
            \begin{enumerate}[label*=\arabic*.]
                \item ¿Qué es un puntero? ¿Cuál es la diferencia entre un puntero y una variable normal?
                \item ¿Qué es un puntero nulo? ¿Para qué se utiliza?
                \item ¿Qué es la aritmética de punteros? Proporcione un ejemplo.
                \item Suponga que $p$ es un puntero de tipo float que almacena la dirección $0x845b342c0$. ¿A qué dirección corresponde el puntero $p+3$?
            \end{enumerate}
            \item Complejidad temporal y espacial
            \begin{enumerate}[label*=\arabic*.]
                \item Ordene las complejidades en notación $\Theta$:
                \begin{multicols}{3}
                \begin{enumerate}
                    \item $n^2$
                    \item $n^3$
                    \item $n$
                    \item $1$
                    \item $\log n$
                    \item $n \log n$
                    \item $n^2 \log n$
                    \item $n^2 + n$
                    \item $n^2 + n^3$
                    \item $n^3 + n^2$
                    \item $n^3 + n$
                    \item $n^3 + n^2 + n$
                \end{enumerate}
                \end{multicols}
                \item ¿Cuál es la complejidad temporal de la siguiente función? Justifique su respuesta.
                \begin{lstlisting}[language=C]
                \end{lstlisting}
                \item ¿Cuál es la complejidad temporal y espacial de la siguiente función? Justifique su respuesta.
                \begin{lstlisting}[language=C]
                \end{lstlisting}

            \end{enumerate}
            \item Recursión
            \begin{enumerate}[label*=\arabic*.]
                \item Escribe una función recursiva en C que calcule el $n$-ésimo término de la secuencia de Fibonacci. La secuencia de Fibonacci se define de la siguiente manera:

                \[
                F(0) = 0, \quad F(1) = 1
                \]
                
                \[
                F(n) = F(n-1) + F(n-2), \quad \text{para } n \geq 2
                \]
                
                El programa debe solicitar al usuario un número entero $n$ y mostrar el $n$-ésimo término de la serie.
                \item Ejecute la función anterior con argumento $n$ igual a 6, indicando todas las llamadas recursivas a la función Fibonacci y que retorna cada llamada.
            \end{enumerate}
        \end{enumerate}
    \end{titlepage}

\end{document}