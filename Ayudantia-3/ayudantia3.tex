\documentclass[12pt]{article}
\usepackage{amsmath} % Necesario para el comando \dfrac
\usepackage{amssymb}
\usepackage{graphicx} % Required for inserting images
\usepackage{enumitem} % Required for customizing enumeration
\usepackage{mathtools} % Necesario para PVI's con funciones por tramos
\usepackage{textcomp}
\usepackage[left=2cm,right=2cm,top=2cm,bottom=2cm]{geometry}
\usepackage[scr]{rsfso}
\usepackage{listingsutf8} % Agrega soporte para utf8 en listings
\usepackage{hyperref}
\usepackage{multicol}
\everymath{\displaystyle}

\lstset{
    language=C,
    inputencoding=utf8,
    extendedchars=true,
    basicstyle=\ttfamily\footnotesize,
    numberstyle=\tiny\color{gray},
    stepnumber=1,
    numbersep=5pt,
    showstringspaces=false,
    breaklines=true,
    tabsize=1,
    literate=
        {ñ}{{\~n}}1
        {Ñ}{{\~N}}1
        {á}{{\'a}}1
        {é}{{\'e}}1
        {í}{{\'i}}1
        {ó}{{\'o}}1
        {ú}{{\'u}}1
        {Á}{{\'A}}1
        {É}{{\'E}}1
        {Í}{{\'Í}}1
        {Ó}{{\'Ó}}1
        {Ú}{{\'Ú}}1
}
\begin{document}

    \begin{titlepage}
        \centering
        \includegraphics[width=0.3\textwidth]{../imgs/logo-uai-fic.png}
        
        \vspace{0.5cm}
        \textbf{\fontsize{12}{24} Ayudantía 3 Estructura de datos}
        
        \vspace{0.5cm}
        \textbf{\fontsize{12}{24}\selectfont Profesores: Sebastián Sáez, Diego Ramos}
        
        \begin{center}
            \textbf{\fontsize{12}{24}\selectfont Ayudantes: Diego Duhalde, Benjamín Wiedmaier, Fernando Zamora}
        \end{center}

        \begin{enumerate}
            % 1
            \item Describa brevemente qué es la búsqueda lineal y la búsqueda binaria, mencione sus principales diferencias y en qué casos es recomendable utilizar cada una.
            
            % 2
            \item Explique en pocas líneas cómo funciona el algoritmo Bubble Sort. Comente cuál es su complejidad temporal en el mejor y peor caso.
            
            %3 
            \item Escriba una función en C que implemente la búsqueda lineal en un arreglo de enteros. La función debe recibir como argumento un arreglo \lstinline[language=C]|array| su tamaño \lstinline[language=C]|largo| y el valor a buscar \lstinline[language=C]|n| y debe retornar el índice donde se encuentra éste o $-1$ en caso contrario.
            
            % 4
            \item Implemente en C la búsqueda binaria para un arreglo de enteros ordenado. La función debe recibir como argumento un arreglo \lstinline[language=C]|array| su tamaño \lstinline[language=C]|largo| y el valor a buscar \lstinline[language=C]|n|  y debe retornar el índice donde se encuentra éste o $-1$ en caso contrario.
            
            % 5
            \item Dado el arreglo $\{ 5, 3, 4, 1, 2 \}$, muestre cada iteración cómo de ordena éste utilizando el algoritmo de Bubble Sort. Indique el estado del arreglo en cada iteración.
            
            % 6
            \item Implemente el algoritmo Bubble Sort en C. Además, busque y aplique una forma de optimizarlo. \textit{\underline{Hint:} Puede utilizar una bandera que perimita detener el algoritmo si durante una iteración no se realizaron intercambios, lo que indica que el arreglo ya está ordenado.}
        \end{enumerate}
    \end{titlepage}
\end{document}