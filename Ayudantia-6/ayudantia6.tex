\documentclass[12pt]{article}
\usepackage{amsmath} % Necesario para el comando \dfrac
\usepackage{amssymb}
\usepackage{graphicx} % Required for inserting images
\usepackage{enumitem} % Required for customizing enumeration
\usepackage{mathtools}
\usepackage{textcomp}
\usepackage[left=2cm,right=2cm,top=2cm,bottom=2cm]{geometry}
\usepackage[scr]{rsfso}
\usepackage{listingsutf8}
\usepackage{hyperref}
\usepackage{multicol}
\everymath{\displaystyle}

\lstset{
    language=C,
    inputencoding=utf8,
    extendedchars=true,
    basicstyle=\ttfamily\footnotesize,
    numberstyle=\tiny\color{gray},
    stepnumber=1,
    numbersep=5pt,
    showstringspaces=false,
    breaklines=true,
    tabsize=1,
    literate=
        {ñ}{{\~n}}1
        {Ñ}{{\~N}}1
        {á}{{\'a}}1
        {é}{{\'e}}1
        {í}{{\'i}}1
        {ó}{{\'o}}1
        {ú}{{\'u}}1
        {Á}{{\'A}}1
        {É}{{\'E}}1
        {Í}{{\'Í}}1
        {Ó}{{\'Ó}}1
        {Ú}{{\'Ú}}1
}
\begin{document}

    \begin{titlepage}
        \centering
        \includegraphics[width=0.3\textwidth]{../imgs/logo-uai-fic.png}
        
        \vspace{0.5cm}
        \textbf{\fontsize{12}{24} Ayudantía 6: Repaso prueba 2}
        
        \vspace{0.5cm}
        \textbf{\fontsize{12}{24}\selectfont Profesores: Sebastián Sáez, Diego Ramos}
        
        \begin{center}
            \textbf{\fontsize{12}{24}\selectfont Ayudantes: Diego Duhalde, Benjamín Wiedmaier, Fernando Zamora}
        \end{center}

        \section*{\raggedright\large Ordenamiento}
        \begin{enumerate}    
            % EJERCICIO 1
            \item Defina qué es un algoritmo de ordenamiento, mencione al menos 3 algoritmos de ordenamiento y explique brevemente cómo funciona cada uno. Además, mencione la complejidad en tiempo de cada uno.
            % EJERCICIO 2
            \item Bajos a trabajar en base al arreglo \texttt{A = [5, 2, 9, 1, 5, 6]}, aplique 3 iteraciones de Selection Sort, 3 iteraciones de Insertion Sort y 3 iteraciones de Bubble Sort. Para cada iteración, indique el arreglo resultante, en el caso de Insertion Sort asuma que la primera iteración parte de i=1.
        \end{enumerate}

        \section*{\raggedright\large Búsqueda}
        \begin{enumerate}    
            % EJERCICIO 1
            \item Describa búsqueda lineal y búsqueda binaria, también mencione la complejidad en el peor caso, mejor caso y caso promedio.
            % EJERCICIO 2
            \item Tenemos el arreglo \texttt{A = [1, 2, 3, 4, 5, 6, 7, 8, 9]}, aplique búsqueda binaria para encontrar el número 3. Describa cada paso de la búsqueda y el resultado final, mencione si estamos más cerca del peor caso, mejor caso o del caso promedio en términos de complejidad.
        \end{enumerate}

        \section*{\raggedright\large Listas y Hashing}
        \begin{enumerate}    
            % EJERCICIO 1
            \item Defina qué es una lista enlazada, mencione sus ventajas y desventajas en comparación con un arreglo. Además, mencione la complejidad en tiempo de las operaciones básicas (\texttt{insertar}, \texttt{eliminar}, \texttt{buscar}).
            % EJERCICIO 2
            \item Tenemos la siguiente lista enlazada: \texttt{A = [1, 2, 3, 4, 5]}, aplique las siguientes operaciones: insertar el número 6 al final de la lista, eliminar el número 2 y buscar el número 4. Describa cada paso de la operación y el resultado final.
            % EJERCICIO 3
            \item Defina qué es una tabla hash, mencione sus ventajas y desventajas en comparación con una lista enlazada. Además, mencione la complejidad en tiempo de las operaciones básicas (\texttt{insertar}, \texttt{eliminar}, \texttt{buscar}).
            % EJERCICIO 4
            \item Suponga que tenemos una tabla hash de tamaño 10 y la siguiente función hash: \texttt{h(x) = x mod 10}. Inserte los siguientes números en la tabla hash: 12, 22, 32, 42, 52. Describa cada paso de la operación y el resultado final. ¿Qué tipo de colisión ocurre? ¿Cómo se podría resolver?
        \end{enumerate}

        \section*{\raggedright\large Colas y Pilas}
        \begin{enumerate}    
            % EJERCICIO 1
            \item Menciona las diferencias principales entre una cola y una pila, da un ejemplo cotidiano de cada una.
            % EJERCICIO 2
            \item Explique en cada caso cómo se puede insertar y eliminar elementos de una cola y una pila. ¿Qué complejidad tienen estas operaciones? ¿Cual opciones nos conviene elegir?
        \end{enumerate}


    \end{titlepage}
\end{document}