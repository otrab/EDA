\documentclass[12pt]{article}
\usepackage{amsmath} % Necesario para el comando \dfrac
\usepackage{amssymb}
\usepackage{graphicx} % Required for inserting images
\usepackage{enumitem} % Required for customizing enumeration
\usepackage{mathtools} % Necesario para PVI's con funciones por tramos
\usepackage{textcomp}
\usepackage[left=2cm,right=2cm,top=2cm,bottom=2cm]{geometry}
\usepackage[scr]{rsfso}
\everymath{\displaystyle}

\begin{document}

\begin{titlepage}
    \centering
    \includegraphics[width=0.3\textwidth]{../imgs/logo-uai-fic.png}

    \vspace{2cm}
    \textbf{\fontsize{12}{24}\selectfont Ayudante: Fernando Andrés Zamora Carrasco}

    \vspace{-1.2cm}
    \textbf{\fontsize{12}{24}\selectfont Profesor: Sebastián Sáez San Juan}

    \vspace{-1.3cm}
    \textbf{\fontsize{12}{24} Ayudantía 1 Estructura de datos}

    \vspace{2cm}


\begin{enumerate}
    % EJERCICIO 1
    \item Responda a las siguientes preguntas sobre punteros.
        \begin{enumerate}[label=\alph*)]
            \item Explique qué es un puntero y cómo se declara en C. Proporcione un ejemplo.
            \item Describa la diferencia entre un puntero y una variable normal. 
            ¿Qué ventajas ofrece el uso de punteros?
            \item Explique qué es la aritmética de punteros y cómo se puede utilizar para recorrer un array.
            \item ¿Qué ees un puntero nulo y cuándo se debe utilizar? 
        \end{enumerate}

    % EJERCICIO 2
    \item Escriba el código apropiado para resolver los siguientes ejercicios
        \begin{enumerate}[label=\alph*)]
            \item Escriba un programa en C que declare un puntero a un entero, asigne un valor a esa variable a través del puntero e imprima el valor y la dirección de memoria.
            \item Escriba un programa en C que utilice punteros para intercambiar los valores de dos variables.
            \item Escriba un programa en C que utilice un puntero para recorrer un array y calcular la suma de sus elementos.
        \end{enumerate}
\end{enumerate}
\end{titlepage}

\newpage
\begin{center}
    \textbf{RESPUESTAS}
\end{center}

\begin{enumerate}
    % RESUPESTA 1
    \item si.
    \item no.
    \item prueba.
\end{enumerate}

\end{document}